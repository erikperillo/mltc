%presentation declaration
\documentclass[11pt]{beamer}
\usetheme{metropolis}

%packages
%beamer stuff
\usepackage{appendixnumberbeamer}
\usepackage{booktabs}
\usepackage[scale=2]{ccicons}
\usepackage{pgfplots}
\usepgfplotslibrary{dateplot}
\usepackage{xspace}
%language
\usepackage[brazilian]{babel}
\usepackage[utf8]{inputenc}
%images
\usepackage{graphicx}
\usepackage{float}

%macros
\newcommand{\img}[2][width=1.0\textwidth]
	{\centering \includegraphics[#1]{img/#2}}
\newcommand{\tit}[1]{\textit{#1}}
\newcommand{\tbf}[1]{\textbf{#1}}
\newcommand{\ttt}[1]{\texttt{#1}}

%title
\title{Classificação de texto}
\subtitle{{\small com Hadoop + Spark Mlib}}
\author{Erik Perillo, RA135582}
\date{}
\institute{Universidade Estadual de Campinas}

\begin{document}

\maketitle

\section{Objetivos}

\begin{frame}{Objetivos}
	Um classificador de textos binário:
	\begin{itemize}
		\item Com uma das melhores técnicas atuais
		\item Com grande volume de dados
	\end{itemize}
\end{frame}

\section{Classificação de Texto}

\begin{frame}{}
	\img[width=1.0\linewidth]{text.png}
\end{frame}

\begin{frame}{Naive Bayes}
	\img[width=0.7\linewidth]{naive.png}
\end{frame}

\section{O projeto}

\begin{frame}{Overview}
	\begin{columns}
		\begin{column}{0.5\textwidth}
			\img[width=0.7\linewidth]{hadoop_logo.png}
		\end{column}
		\begin{column}{0.5\textwidth}
			\img[width=0.7\linewidth]{spark_logo.png}
		\end{column}
	\end{columns}
\end{frame}

\begin{frame}{Apache Spark}
	\begin{columns}
		\begin{column}{0.5\textwidth}
			\begin{itemize}
				\item Computação em \tit{batches} de alto desempenho
				\item Análise de dados avançada
				\item Processamento de \tit{stream} em tempo real
				\item Interfaces para Java, Scala, Python
			\end{itemize}
		\end{column}
		\begin{column}{0.5\textwidth}
			\img[width=0.7\linewidth]{spark_logo.png}
		\end{column}
	\end{columns}
\end{frame}

\begin{frame}{Spark: componentes}
	\img[width=0.5\linewidth]{spark_stack.png}

	\img[width=0.5\linewidth]{hadoop_logo.png}
\end{frame}

\begin{frame}{Spark Mlib}
	\begin{columns}
		\begin{column}{0.5\textwidth}
			\begin{itemize}
				\item Diversas técnicas de \tit{Machine Learning}
				\item Desenvolvimento rápido e escalável
			\end{itemize}
		\end{column}
		\begin{column}{0.5\textwidth}
			\img[width=0.7\linewidth]{spark_mlib.png}
		\end{column}
	\end{columns}
\end{frame}

\section{Juntando tudo}

\begin{frame}{Juntando tudo}
	\begin{itemize}
		\item Tudo foi feito em Python :)
		\item Foi implementada visualização pela linha da comando
	\end{itemize}
\end{frame}

\begin{frame}{Flow dos dados}
	\begin{itemize}
		\item Arquivos de uma database entram (\ttt{20\_newsgroup})
		\item Pré-processamento do texto é feito (\tit{stemming, stopwords...})
		\item Conversão de texto para vetor de \tit{features} (\ttt{if-idf})
		\item Textos positivos/negativos são convertidos para .csv
		\item Arquivo .csv é convertido em formato para \tit{Spark}
		\item Treinamento é feito com \tit{Naive Bayes}
	\end{itemize}
\end{frame}

\section{Resultados}

\end{document}
